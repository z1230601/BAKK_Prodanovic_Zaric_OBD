%%%% Time-stamp: <2012-08-20 17:41:39 vk>

%% example text content
%% scrartcl and scrreprt starts with section, subsection, subsubsection, ...
%% scrbook starts with part (optional), chapter, section, ...

\chapter{On Board Diagnosis Basics}
\label{sec:OBDBASICS}
OBD's original purpose was to monitor CO2 emissions for climatic reasons, but since its introduction its range of abilities 
extended. Rather than just monitoring CO2 emissions the surveillance of a majority of the sensors inside an automobile as well as 
keeping track of the error memory by trouble codes is possible. Even the access of the bus gateway and therefore all its busses and 
their attached entities is feasible, resulting in a not negligible security threat. 

These characteristics are a result of introducing the concept of a central OBD interface at a time, 
when almost every automotive manufacturer already implemented own communication protocol between the separate control units [\footcite{SCHAFOBD1}].
As a result the OBD standard does not define a specific protocol used for the CAN bus but only the protocol used to communicate with the diagnostic 
section of a control unit, where standardized trouble codes can be obtained. 

If the protocol used for inter CU communication was known, it would be possible to send commands directly to any CU in the bus system, which bears a 
great security issue. This issue was one of the most important disadvantages critics discussed since the introduction of OBD, but due to the
popularity of this topic at universities this thesis will not cover it [\footcite{Koscher2010}].

Generally there is a variety of already existing protocols and standards which define the communication via the OBD interface. The usage of the ELM327 
emerged due to its availability and simplicity, as it is a popular programmed microcontroller for translating the OBD interface. ELM327 has the 
advantage of establishing a connection between the OBD interface and a PC using different standards and protocols e.g. SAE J1850, ISO 14230-4 KWP, 
ISO 15765-4 CAN. Basically the ELM327 serves as an interface communicating as a UART (universal asynchronous receiver/transmitter) device with the 
PC it is attached to and K-Line to communicate with the CU. The hardware port variates depending on the intended usage and circumstances and can go 
from USB over Bluetooth to Wi-Fi or even RS-232 [\footcite[pp. 46 ff.]{SCHAFOBD2}].

\setcounter{section}{0}
\section{Busses}

Busses are an essential part of every computer system. As this project covers the communication to external hardware, which in turn communicates to other 
external hardware, busses are an important point for the completion of this thesis. This chapter’s focus lies on the busses used by the ELM327 and more 
specific to the busses to which basic knowledge is recommended for this project.

The automotive industry applies KL-Line bus, MOST bus as well as CAN. As the busses used for diagnosis are abstracted due to the usage of the ELM327, 
this chapter will not cover the CAN, MOST and KL-Line bus but instead it will provide basics on USB which are fundamental for \nameref{sec:emulation} 
and \nameref{sec:serialComm}.

\subsection{Universal Serial Bus (USB)}

This chapter introduces the relevant USB basics regarding the project requirements gathered from \cite{USBNUT} which summarizes the official \citetitle{USB} 
[\footcite[pp. 36ff., 199ff., 260-274]{USB}].

USB is used for a vast range of devices. As the project requirements for the selection and communication to as well as the emulation of an USB device,
it seems appropriate to cover the relevant USB basics. Readers who already have experience with USB or knowledge about the general functionality of the 
protocol can skip this chapter.

USB is a shielded 4 wire, TTL voltage communications interface using NRZI (Non Return to Zero Inverted) encoding. Meaning that it is represented by 4 wires:

\begin{itemize}
\item High (5V reference Voltage)
\item D-
\item D+
\item Ground (reference Voltage)
\end{itemize}

\emph{D} is a twisted pair, differential signaling wire. Differential signaling is a method to reduce noise effects on data transmissions where a signal
is transmitted over the wires, with the additional property that the signal carried by one wire is inverted to the signal carried by the other wire. The 
receiver uses the difference between the signals instead of the signal itself bracing it against overlaid noise [\footcite{Massoud2001}].

It has a tiered star topology which means that although they are physically cascaded, e.g. by hubs, it is logically considered a star topology where the 
root hub (computer) can access a device over its logical address (port).
\begin{figure}

\end{figure}

Since it is a host centric communication every transaction is initiated by the host. Each transaction has a header packet (token), an optional data 
packet and a handshake packet. The header packet defines targets (interface and endpoint) and the handshake represents the validity of the response, 
whereas the data packet is simply carrying the payload.

Token packets can either be IN, OUT or SETUP. IN signals the USB device that the host is ready to read data from it, OUT that it is ready to write data 
to it and SETUP is used as an identifier for control transfers, which are instruction transfers to the device itself.

Data packets define their states as DATA0, DATA1, DATA2 and MDATA, whereas the latter two are only available in high speed mode. Each can transfer a 
maximum of 1024 bytes of data depending on the speed mode.

Finally the handshake packet can take the values ACK, NAK, STALL, which signals the host that the transaction was successful, unsuccessful or that the 
device is waiting for the host to initiate a transaction.%\footcite[p. 199ff.]{USB}

\subsubsection{4 Transfer Types}

Although USB being a host centric communication the standard supplies 4 different types of transfers: 

\begin{itemize}
 \item Control
 \item Interrupt
 \item Isochronous
 \item Bulk
\end{itemize}

Generally they can be divided into two types of transfers: control and data. Control transfers are communications with the purpose of changing the 
configuration of the device. Its functions include getting and setting selectable languages, setting up USB specific attributes, such as hub and port 
addresses, and selecting configurations. Typically those kinds of transfers are small in size and amount. 
The term data transfers sums up the latter three transfers in the above list. They are transfers transmitting data to the functionality 
behind the interface.%\footcite[p. 36ff.]{USB}

\paragraph{Control Transfers}

Each control transfer can undergo three possible stages: 

\begin{description}
 \item[Setup Stage] Different Settings/Data are requested
 \item[Data Stage] Requested data is sent
 \item[Status Stage] Handshake and confirmation of successful transmission takes place
\end{description}

As an example one can imagine a digital camera, which can either be used as a webcam or as a storage device to get the data already 
memorized on it. When attached, the host only knows that an USB device was attached. It has to request the different endpoints for the 
different functionalities from the device. Control transfers will be initiated to request the configuration acquiring those settings 
(setup stage). The transmission of the descriptor hierarchy (see \nameref{sec:descriptors}), which represents the different functionalities and the 
corresponding ports, will take place in the data stage.

Successful transfers will be ensured by transmitting another set of packets with zero length to confirm its validity.

\paragraph{Interrupt Transfers}

As almost any hardware device, USB devices have the capability to send interrupts as well. As mentioned before USB is a host centric communication 
meaning that, although being an interrupt transfer, it still needs to be polled by the host. 

\paragraph{Isochronous Transfers}

Isochronous transfers are specified for data which is streamable, like video or audio. It has the specialty of not requesting a resend of data if a 
packet is not transmitted correctly. The maximum data payload size, depending on the type of communication, is up to 1024 bytes and is defined in 
the endpoint descriptor (see \nameref{sec:descriptors}). It is common practice to define multiple interfaces for isochronous endpoints as in case of 
bandwidth restrictions alternatives can be provided.

\paragraph{Bulk Transfers}
As the name suggests bulk transfers are meant to deal with bulky data, like files or images. It does not guarantee a latency or 
bandwidth. Therefore it offers a secure way of transmitting data as retransmission is initiated on failed transfers. Unfortunately 
only devices supporting full or high speed can use this type of transfer. 

\subsubsection{Descriptors}
\label{sec:descriptors}
Due to the necessity of transmitting USB configurations to the host, different descriptors defining the devices functionality and its coupling to
the endpoints exist. This subchapter shall give an overview of these descriptors in means of type and structure since it is an essential part of the 
emulation. 

In the simplest terms descriptors are data representing or describing a device as a structure. There are different kinds of descriptors for different 
purposes. The following are defined in the USB standard:

\begin{description}
 \item[Device Descriptor] Every device can only have one device descriptor.
 \item[Configuration Descriptor] Each device descriptor has one or more configuration descriptors.
 \item[Interface Descriptor] Each configuration descriptor has one or more interface descriptors.
 \item[Endpoint  Descriptor] Each interface descriptor has two or a multiple of two endpoint descriptors.
 \item[String Descriptor] Each descriptor can have zero or more string descriptors.
\end{description}
Each descriptor contains the summed length, including the length itself, as the first argument and the descriptor type as the second one. There is 
always at least one device, configuration, interface and endpoint descriptor.

\paragraph{Device Descriptor}
The device descriptor serves as general information about the device. It contains the vendor and product id identifying the exact type of device as 
well as the binary coded decimal device code defined in the official USB standard. We used the feature of setting this device code to 255, which will 
induce the behaviour of vendor specific driver loading. That being said, if a vendor should be a so called FTDI device, such as microchips, the 
driver for serial communications are loaded. Relevant elements of the device descriptor are:%\footcite[p. 260ff.]{USB}

\begin{table}[h!]
\begin{tabular}{c| p{10cm}}
Field & Description \\ \hline
bLength & length of the complete packet (including the length itself) in bytes \\
bDescriptorType & type of descriptor (device 0x01) \\
bcdUSB & corresponding USB version number in (binary coded decimal) \\ 
bMaxPacketSize & maximum packet size (either 8, 16, 32, 64)\\
idVendor & vendor id of the device\\
idProduct & product id of the device\\ 
bcdDevice & corresponding device integer (binary coded decimal) \\
iManufacturer & ID of the string descriptor representing the manufacturer\\
iProduct & ID of the string descriptor representing the product \\
bNumConfigurations & amount of configurations \\ 
\end{tabular}
\caption{Device Descriptor Fields}
%\footcite[p. 262\-263]{USB}
\end{table}

\begin{verbatim}
 const uint8_t dev_desc[] = {
	18,     // descriptor length
	1,      // type: device descriptor
	0x00,   // bcd usb release number
	0x02,   //  "
	0,      // device class: per interface
	0,      // device sub class
	0,      // device protocol
	8,     // max packet size
	0x03,   // vendor id
	0x04,   //  "
	0x01,   // product id
	0x60,   //  "
	0x00,   // bcd device release number
	0x06,   //  "
	1,      // manufacturer string
	2,      // product string
	3,      // serial number string
	1       // number of configurations
};
\end{verbatim}

\paragraph{Configuration Descriptor}
A USB device can provide different configurations which depend on the power mode of the attached device. The configuration descriptor supplies mostly 
power settings. Self powered mode indicates that an external power source is applied. It can be chosen by devices with a limited amount of power. On 
the other hand bus powered settings will require more energy to be transmitted over the bus thus draining more energy of the host device. Additionally 
configuration descriptors serve as a grouping of interfaces.%\footcite[p. 264ff.]{USB}
\begin{table}[h!]
\begin{tabular}{c|p{10 cm}} 
Field & Description \\ \hline
bLength & complete length of the packet (include the length itself) in bytes \\
bDescriptorType & type of descriptor (device 0x02)\\
wTotalLength & length of the returned data \\
bNumInterfaces & amount of interfaces\\
bConfigurationValue & index of configuration \\
iConfiguration & ID of the string for the configuration name\\
bmAttributes  power mode\\
bMaxPower & the maximum power consumption (consumption = value * 2mA) \\
\end{tabular}
 \caption{Configuration Descriptor Fields}
 %\footcite[p. 265\-266]{USB}}
\end{table}


\paragraph{Interface Descriptor}
Interface descriptors have the purpose of representing a single functionality and grouping the needed endpoints together, thus making 
it simpler for the host to select a specific functionality.%\footcite[p. 267ff.]{USB}

\begin{table}
\begin{tabular}{c|p{10 cm}}
Field & Description \\
bLength &  complete length of the packet (include the length itself) in bytes \\
bDescriptorType &  type of descriptor (device 0x04) \\
bInterfaceNumber & index of interface \\
bAlternateSetting & index of alternative interface if this should fail due to e.g. speed capabilities \\
bNumEndpoints & amount of endpoints \\
bInterfaceClass & standardized class id (in USB standard) describing the interface type \\
bInterfaceSubClass & standardized class id (in USB standard) describing the interface type \\
bInterfaceProtocol & standardized protocol id (in USB standard) describing the protocol type \\
iInterface & ID of string for interface name (string descriptor request) \\
\end{tabular}
 \caption{Interface Descriptor Fields}
 %\footcite[p. 268\-269]{USB}}
\end{table}

\paragraph{Endpoint Descriptor} 

Endpoints are the sinks and sources of data for the device. As mentioned above different types of transfers exist, hence different types of 
endpoints can be defined. Endpoint descriptors are necessary to define those endpoints.%\footcite[p. 269ff.]{USB}

\begin{table}
\begin{tabular}{c|p{10 cm}}
Field & Description \\ \hline
bLength & complete length of the packet (include the length itself) in bytes \\
bDescriptorType & type of descriptor (device 0x05) \\
bEndpointAddress & address of the endpoint and direction\\
bmAttributes & type of endpoint (see transfer types)\\
wMaxPacketSize & max receiving or sending endpoint size\\
bInterval & for interrupt transfer type endpoints define polling rate in frames\\
\end{tabular}
 \caption{Endpoint Descriptor Fields}
 %\footcite[p. 269\-271]{USB}}
\end{table}

\paragraph{String  Descriptor}
Since the descriptors need some string with variable length, instead of defining a maximum byte length for string fields in descriptors and thus 
limiting the developers freedom by a specific char amount, the USB standard defines an additional descriptor type. Strings are 
described by a, in the device context, unique ID and can be separately requested in  different languages. Therefore string descriptors 
are utile descriptors whose purpose is to enable free name definition and also offer multilanguage support.%\footcite[p. 273ff.]{USB}

\begin{table}
\begin{tabular}{c|p{10 cm}}
Field & Description \\ \hline
bLength & complete length of the packet (include the length itself) in bytes \\
bDescriptorType & type of descriptor (device 0x03)
\end{tabular}
 \caption{String Descriptor Main Fields}
 %\footcite[p. 273\-274]{USB}}
\end{table}

In case of string descriptor ID 0 the selectable languages are returned as a list of standardized IDs
\begin{table}
\begin{tabular}{c|p{10 cm}}
wLANGID[0] & first language id in form of (0xXXXX) \\
wLANGID[1] & second language \\
wLANGID[x] & n language 
\end{tabular}
 \caption{String Descriptor Language Fields}
 %\footcite[p. 273]{USB}}
\end{table}

In case of any other string descriptor ID the message would end as following:
\begin{table}
\begin{tabular}{c|p{10 cm}}
bString & unicode string where every single char is separated by a null byte
\end{tabular}
 \caption{String Descriptor Fields}
 %\footcite[p. 274]{USB}}
\end{table}

\begin{verbatim}
 uint8_t conf_desc[] = {
	9,      // descriptor length
	2,      // type: configuration descriptor
	32,     // total descriptor length (configuration+interface)
	0,      //  "
	1,      // number of interfaces
	1,      // configuration index
	0,      // configuration string
	0xa0,   // attributes: none
	45,      // max power

	9,      // descriptor length
	4,      // type: interface
	0,      // interface number
	0,      // alternate setting
	2,      // number of endpoints
	0xFF,      // interface class
	0xFF,      // interface sub class
	0xFF,      // interface protocol
	2,       // interface string
	// endpoint 1 in
	7,      //bLength
	5,      //desriptor type
	0x81,    //ep in
	2,         // attributes
	0x40,    //maxpacketsize
	0x00,    //"
	0,      // intervall
	//endpoint 2 out
	7,      //bLength
	5,      //desriptor type
	0x2,    //ep in
	2,         // attributes
	0x40,    //maxpacketsize
	0x00,    //"
	0      // intervall
}; 
\end{verbatim}

\subsubsection{Requests}

The previous sections introduce different types of transfers. One of them is the control transfer which serves as a way of controlling 
the USB device. There are requests which each device is required to implement due to the specifications of the USB standard. Basically 
the use of a request can be derived by its name. The existing requests are listed below.%\footcite[p. 267ff.]{USB}

\paragraph{Device Requests}
\begin{itemize}
 \item GET\_STATUS
 \item CLEAR\_FEATURE
 \item SET\_FEATURE
 \item SET\_ADDRESS
 \item GET\_DESCRIPTOR
 \item SET\_DESCRIPOTR
 \item GET\_CONFIGURATION
 \item SET\_CONFIGURATION
\end{itemize}

\paragraph{Interface Requests}
\begin{itemize}
 \item GET\_STATUS
 \item CLEAR\_FEATURE
 \item SET\_FEATURE
 \item GET\_INTERFACE
 \item SET\_INTERFACE
\end{itemize}

\paragraph{Endpoint Requests}
\begin{itemize}
 \item GET\_STATUS
 \item CLEAR\_FEATURE
 \item SET\_FEATURE
 \item SYNC\_FRAME
\end{itemize}

Each of these requests is identified by a combination of two ID fields called request type and request. Each request follows the 
following structure:

\begin{itemize}
 \item bmRequestType (1 byte)
 \item bRequest (1 byte)
 \item wValue (2 bytes)
 \item wIndex (2 bytes)
 \item wLength (2 bytes)
\end{itemize}

\subsection{Command Structure and Behaviour}

At the beginning of automotive manufacturing almost all components were mechanical. As technology developed more and more electrical 
components found their way into the automobile. At some point the requirement for a standardized interface emerged. Due to the fact 
that different manufacturers had already applied different technologies, the problem of standardizing their systems would have 
resulted in huge redevelopment costs. As no manufacturer could be expected to carry those costs, the OBD standard does not define the 
utilization of any specific technology. Instead it defines the interface and format to which every automobile must be able to 
communicate with standardized commands [\footcite{SCHAFOBD1}].

This diagnostic data must have a uniform format. This format is called diagnostic trouble code or simply fault code. Since every 
vehicle has the same theoretical background, standardizing types of faults is feasible. This property also simplifies a categorization 
of trouble codes, whereby different standards for diagnosing vehicles have been introduced by the International Organization for 
Standardization (ISO) and Society of Automotive Engineers (SAE). 

Due to the amount of integrated control units, their varying function set and size as well as their location a need for clustering is 
displayed. However, they commonly can be classified into 4 categories depending on their functionality in the vehicle:
\begin{itemize}
 \item Powertrain 
 \item Chassis 
 \item Body 
 \item Network
\end{itemize}

\section{Service Modes}

The SAE J1979 and the ISO 15031-5 are two standards which define the available service modes needed to make a diagnostic analysis of cropped up fault codes. Currently there are ten prevalent operation modes, but there are still five reserved modes for eventual future modifications of the actual standard. Additionally manufacturers have the possibility to implement their own diagnoses by using higher service modes.
Nine of the ten specified modes provide information of sensor measurements or triggered diagnostic trouble codes recorded in the error memory. The remaining mode is meant to clear the memory. Nowadays automobiles are not obligated to support every service  mode, de facto just a limited amount of information can be acquired via the OBD interface.
Following the ten service modes as well as a short description of their respective function are listed [\footcite{SCHAFOBD2}]:

\begin{description}
 \item[Request current data] depending on the sensor availability actual measurements can be requested as well as the latest recordings of diagnostic trouble codes.
 \item[Request freeze frame data] faults, causing the malfunction indicator lamp (MIL) to flash while the vehicle is in motion, get recorded in a memory area which is the so called freeze frame. More precisely the diagnostic data, requestable by service mode one, at the moment of the fault occurrence is memorized. Enabling a more detailed read out and debugging of the error cause.
 \item[Request stored diagnostic trouble codes] this mode is used to read out emission based faults without deleting the fault codes from memory.
 \item[Delete stored diagnostic trouble codes] this service mode provides the capability of deleting all stored DTCs from the memory irreversibly. Furthermore the freeze frame data as well as the readiness code, which contains the data of the continuous and non-continuous monitoring of the car’s self-testing emission system, is reset. It is recommended to use this mode cautiously, as it also resets the MIL and all its related counters.
 \item[Request O2-Sensor values] this mode ascertains the measured values of the O2-sensor. The conversion of the requested values is depending on the manufacturer.
 \item[Request specific, non-continuous monitored systems] similar to service mode five the sixth service mode especially provides data of non-continuous monitored components. The range of considered components is  manufacturer dependent.
 \item[Request pending DTCs] this mode allows a comfortable way to read out reoccurring fault codes during a test drive after the vehicle got repaired and the DTCs got deleted. Especially fault codes which do not show up in service mode three can be emphasized with mode seven.
 \item[Control on-board systems/test/components]  this mode supplies external control of on-board systems, testing functions or other components by turning them temporarily or permanently  on or off.
 \item[Request vehicle information] the vehicle information contains e.g. the VIN (vehicle identification number) as well as several calibration values.
 \item[Permanent emission-based DTCs] this service mode is only available to CUs which communicate via the CAN protocol. DTCs stored in this part of the memory cannot be cleared by service mode four due to their severeness, instead only several road cycles will cause the CU to clear the fault if the problem does not reappear.
\end{description}

Furthermore so called OBD PIDs (Parameter Identifiers) are needed to be able to request data from a vehicle successfully. In the following section 
we discuss the characteristics of PIDs [\footcite{SCHAFOBD1}].

\subsection{Parameter Identifier}
Parameter identifier are used to specify the exact sub functionality or location of the selected service mode. A sub functionality can
be a specific sensor or requested data like VIN. Every service mode contains an own set of available, valid PIDs to produce meaningful requests. Many PIDs are 
already specified and included in the OBD-II standard SAE J1979. Due to the fact that each manufacturer can define vehicle specific PIDs it is not 
guaranteed that all vehicles will respond to all available PIDs, even though they are valid. Leading to the solution of reserving the first PID in a 
set of 32 PIDs for requesting the valid PIDs of the set. Depending on the service mode in combination with the parameter identifier the byte-length 
as well as the value-range of the response can vary. 

%% vim:foldmethod=expr
%% vim:fde=getline(v\:lnum)=~'^%%%%\ .\\+'?'>1'\:'='
%%% Local Variables: 
%%% mode: latex
%%% mode: auto-fill
%%% mode: flyspell
%%% eval: (ispell-change-dictionary "en_US")
%%% TeX-master: "main"
%%% End: 

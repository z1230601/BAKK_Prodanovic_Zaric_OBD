%%%% Time-stamp: <2013-02-25 10:31:01 vk>


\chapter*{Abstract}
\label{cha:abstract}

Considering the increasing amount of control units in automotive environments an effective handling of these units seems inevitable. Due to the secrecy 
of automotive manufacturers common service mechanics are no longer independently capable of maintaining all components of a vehicle. Furthermore wrong assumptions of the 
maintanence personnel often lead to incorrect error identification and thus the costly replacement of fully functional parts. A solution is offered by 
the ISO$15031$ defining a unique way of identifying faults with the on board diagnostics (OBD) interface.

This thesis analyses the capabilites of OBD in a Linux environment, to produce open source software, thus reducing dependencies on automotive manufacturers. 
Therefore it offers relief to hobby mechanics due to its open source property. Furthermore this thesis provides a base package including all essential 
functionalities of OBD. As the development utilizes the ELM microcontroller, this software is compatible with this low cost hardware solution for OBD functionality. 
In the development phase an analysis of the sourrounding development process is conducted, experiencing the combination of \citetitle{AUTOSPICE} with agile software 
development. Furthermore the encountered issues and correlations of this newly created development process is portrayed.

The motives behind this thesis are located in the chapter \nameref{sec:motivation} as well as an market analysis building a foundation for it. After establishing the 
need for this kind of software the document continues with a theoretical basis layed out in the chapter \nameref{sec:OBDBASICS}. The realization of this theory can 
be looked up in the chapters \nameref{sec:impl} and \nameref{sec:projectspecific}. The chapter \nameref{sec:SDT}, which inputs theory and 
execution of development processes, along with the \nameref{sec:summary} chapter concludes this thesis and gives an perspective on future possibilities.

%\glsresetall %% all glossary entries should be used in long form (again)
%% vim:foldmethod=expr
%% vim:fde=getline(v\:lnum)=~'^%%%%\ .\\+'?'>1'\:'='
%%% Local Variables:
%%% mode: latex
%%% mode: auto-fill
%%% mode: flyspell
%%% eval: (ispell-change-dictionary "en_US")
%%% TeX-master: "main"
%%% End:

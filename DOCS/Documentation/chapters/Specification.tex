%%%% Time-stamp: <2012-08-20 17:41:39 vk>

%% example text content
%% scrartcl and scrreprt starts with section, subsection, subsubsection, ...
%% scrbook starts with part (optional), chapter, section, ...

\chapter{Specification}
The last chapters established that Linux based systems offer more possibilities and a lack of OBD software solutions making it the base platform for 
this project. Another discovery discussed in the previous chapter is the division of the project into three major parts. The theoretical analysis of 
OBD, the OBD test tool, more precisely the ECU emulation and the OBD tool itself. Since this approach bears the danger of code duplication and 
decentralized functionalities another optimization is required.  Extracting the similar functionalities of the latter two project parts leads to a 
commonly used OBD middleware part. 
The OBD middleware is located at the root of the project and consists of:
\begin{description}
\item[Device Communication]

This entails the communication with the USB device as well as the emulation of the ELM for the OBD test tool.
\item[Configuration Management]

To keep this project as flexible as possible it will use a lot of configuration files. Therefore an object representation of those configuration 
classes will be necessary.
\item[Database Management]

The amount of currently existing diagnostic trouble codes (DTC’s) is vast, therefore a database responsible for their management is a reasonable 
solution.
\item[OBD Base Elements]

The representation of the DTC and their responses are coupled in this package as well as the ELM command and the different types of OBD values.
\end{description}

The character of the OBD middleware makes this package a perfect starting point for this project. 
The different project parts are explained in the chapter Implementation and Project Specific Implementation

\setcounter{subsection}{0}
\subsection{Technical Specifications}
Due to the increasing popularity of agile software development and consistency of using Automotive SPICE in automotive systems 
an additional scientific part arose. The combination of  the  widespread technique of agile software development with the  rigid Automotive 
SPICE concept additionally makes the implementation of the project a case study on software development in the automotive context. 
The experiences and compromises are described in chapter Software Development Techniques  in detail. 

%% vim:foldmethod=expr
%% vim:fde=getline(v\:lnum)=~'^%%%%\ .\\+'?'>1'\:'='
%%% Local Variables: 
%%% mode: latex
%%% mode: auto-fill
%%% mode: flyspell
%%% eval: (ispell-change-dictionary "en_US")
%%% TeX-master: "main"
%%% End: 

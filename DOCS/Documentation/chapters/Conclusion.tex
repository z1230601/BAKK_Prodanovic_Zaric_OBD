%%%% Time-stamp: <2012-08-20 17:41:39 vk>

%% example text content
%% scrartcl and scrreprt starts with section, subsection, subsubsection, ...
%% scrbook starts with part (optional), chapter, section, ...

\chapter{Conclusion}
\label{sec:conclusion}

The combination of the spontaneous agile software development with the rigid Automotive SPICE is fusing two contradictory software development areas. 
This results in numerous issues and gathered experience which will be shared in this chapter.

Right after the SPICE PRM planning phase was finished, the actual work on the tools started. The first achievable goal was to establish a 
running communication via the serial interface, more precisely connecting over the USB interface. Due to the early stage of development, more 
accurately the lack of practical experience with the newly created methodology, the implementation took place without real unit tests. 
Furthermore meaningful testing would have been complicated as well, due to the necessity of implementing mock classes to successfully test 
all parts of the serial communication.

After completing the serial communication the development of the OBD related parts were planned. Whilst realizing this part and planning the next 
huge overlaps emerged. In accordance to agile software development practices common functionalities of both, OBDCU and OBD tool, requirements were 
encapsuled in an OBD middleware package. Its purpose is to provide classes and functions which are needed by both softwares. Therefore changing the 
plan acquired in previous PRM processes, which is a good comparison between agile and rigid software development.

As time went by adjustments to the characteristics of agile software development were made, meaning the writing of unit tests right before actual 
production code was practiced constantly as well as regularly switching the roles as suggested for pair programming. While one wrote an unit 
test which failed after its immediate execution the other one had to observe and point out possible issues. After the unit test was finished, 
the developer of the unit test switched into the observer role and the other one implemented a solution to pass the unit test as well as going 
for the next unit test, alternating the roles constantly.

Over time the quality of the unit tests improved and furthermore the exertion of pair programming led to less debugging and refactoring compared 
to previous non agile projects.

Besides the decision of developing a middleware for the OBD software the basic principle of structuring retained as planned from the beginning 
on. Unfortunately cuts in the final softwares were made due to underestimating the time which was consumed by the general implementation and 
refactoring. This leaves space for continous

%% vim:foldmethod=expr
%% vim:fde=getline(v\:lnum)=~'^%%%%\ .\\+'?'>1'\:'='
%%% Local Variables: 
%%% mode: latex
%%% mode: auto-fill
%%% mode: flyspell
%%% eval: (ispell-change-dictionary "en_US")
%%% TeX-master: "main"
%%% End: 

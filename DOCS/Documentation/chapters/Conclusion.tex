%%%% Time-stamp: <2012-08-20 17:41:39 vk>

%% example text content
%% scrartcl and scrreprt starts with section, subsection, subsubsection, ...
%% scrbook starts with part (optional), chapter, section, ...

\chapter{Summary}
\label{sec:summary}
The combination of the spontaneous agile software development with the rigid Automotive SPICE is fusing two contradictory software development areas. 
This results in numerous issues and gathered experience which will be shared in this chapter.

\section{Conclusion}
\label{sec:conclusion}

Right after the SPICE PRM planning phase was finished, the actual work on the tools started. The first achievable goal was to establish a 
running communication via the serial interface, more precisely connecting over the USB interface. Due to the early stage of development, more 
accurately the lack of practical experience with the newly created methodology, the implementation took place without real unit tests. 
Furthermore meaningful testing would have been complicated as well, due to the necessity of implementing mock classes to successfully test 
all parts of the serial communication.

After completing the serial communication the development of the OBD related parts were planned. Whilst realizing this part and planning the next 
huge overlaps emerged. In accordance to agile software development practices common functionalities of both, OBDCU and OBD tool, requirements were 
encapsuled in an OBD middleware package. Its purpose is to provide classes and functions which are needed by both softwares. Therefore changing the 
plan acquired in previous PRM processes, which is a good comparison between agile and rigid software development.

As time went by adjustments to the characteristics of agile software development were made, meaning the writing of unit tests right before actual 
production code was practiced constantly as well as regularly switching the roles as suggested for pair programming. While one wrote an unit 
test which failed after its immediate execution the other one had to observe and point out possible issues. After the unit test was finished, 
the developer of the unit test switched into the observer role and the other one implemented a solution to pass the unit test as well as going 
for the next unit test, alternating the roles constantly.

Over time the quality of the unit tests improved and furthermore the exertion of pair programming led to less debugging and refactoring compared 
to previous non agile projects.

Besides the decision of developing a middleware for the OBD software the basic principle of structuring retained as planned from the beginning 
on. Unfortunately cuts in the final softwares were made due to underestimating the time which was consumed by the general implementation and 
refactoring. This leaves space for further functionality improvements.

\section{Outlook}
\label{sec:outlook}

The thesis started with the intention of researching and implementing the OBD technology, as explained in \namere{sec:motivation}. After consulting 
the thesis' advisor, Phd. Harald Sporer, the assumption that the workload does not suffice for a bacheloar thesis arose. In the following meetings 
the concept of implementing a testing tool for OBD tools as well emerged, which was accepted by all parties. The actual workload was gravely 
underestimated at that time.

As described in \nameref{sec:conclusion} structural changes were made during the implementation phase. On top of that OBD capabilites 
offer far more possibilities than exhausted by this thesis. Therefore this chapter describes the next feasible steps for further improvements.

Accomplishing the full sensor capabilites of the OBDCU by implementing the lacking user interface is a recommendable starting point. 
As described in \nameref{sec:obdbase} the back-end functionality already exists. Furthermore these sensor functions need to be integrated into the system, by
loading the right configuration as well as potentially expanding the interaction with other parts like trouble codes. 

The adaption of the OBDCU to the OBDTool is another potential task. This entails the refactoring of the graphical user interface to the OBD tool's specific 
requirements, by using different parts of the OBDMiddleware package. That includes the usage of the serial communication instead of the emulation, the adaption of the sensor data 
as well as the diagnostic trouble code feature from output to input. The database manipulation does not require any changes. 

Finally, further opportunities can be found in the CAN communnication abilities of the ELM microcontroller. It can directly communicate to the diagnositc control unit and 
in turn the bus gateway control unit, enabling CAN command broadcasts to any CAN node in the car. Since a majority of control signals use the CAN Bus interface it offers partially 
security sensitive features. When considering the expansion of computer science in the automotive industry, like self driving cars, and secrecy concerning restricted IT background knowledge of 
automotive manufacturers, it is safe to state that the importance of interfaces like OBD and moreover the need for software described in this thesis will increase in the near future.  

%% vim:foldmethod=expr
%% vim:fde=getline(v\:lnum)=~'^%%%%\ .\\+'?'>1'\:'='
%%% Local Variables: 
%%% mode: latex
%%% mode: auto-fill
%%% mode: flyspell
%%% eval: (ispell-change-dictionary "en_US")
%%% TeX-master: "main"
%%% End: 

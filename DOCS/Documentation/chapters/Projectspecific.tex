%%%% Time-stamp: <2012-08-20 17:41:39 vk>

%% example text content
%% scrartcl and scrreprt starts with section, subsection, subsubsection, ...
%% scrbook starts with part (optional), chapter, section, ...

\chapter{Project Specific Implementation}

The preceding package implementation descriptions only focus on overlapping functionalities. 
The result is the OBDMiddleware package which is an API.
Furthermore a concrete application has to be implemented to apply the designed features.
The OBDCU, which stands for On Board Diagnosis Control Unit, represents the emulated diagnosis CU.
It offers possibilities to emulate attached USB devices on Linux systems and is able to set and unset
permanent and pending diagnostic trouble codes as well as deleting them. 
The OBD tool was supposed to be its counterpart for reading trouble codes and its corresponding freeze frame data,
but since our time contingent did not suffice the OBD tool as well as the OBDCU’s sensor capabilities were not implemented. 
Although it is safe to say that the general functionalities for requesting sensor data is included in the OBDMiddleware. 
Even though the common mechanics are asserted by unit tests,
the additional implementation and connection of the related sensor functions would easily overstep the efforts of this bachelor thesis.

CMake ensures easy compilation and first and foremost dependencies to external packages
in combination with the usage of self implemented packages. 
A setup for development was provided by making paths to the subprojects configurable. 
Packaging for installation was ensured by using CPack generators, especially for Debian systems.

The Model-View-Controller architecture was chosen due to its clear specification of classes and their clustering into general divisions. The model being the middleware package itself.

\section{OBDCU}

The OBDCU application offers the user the possibility to emulate and partially configure an USB FTDI device with one configuration
and one set of bulk endpoints. Furthermore diagnostic trouble codes are selectable from a database and settable to the emulated 
fault memory either as a permanent or temporary trouble code. 
Additionally it supplies a surface to configure responses to ELM commands (e.g. atz). 
While the current version only supplies a beta version with several restrictions regarding the configurations, 
since they are only saved over the runtime of the application. 
However it is able to simulate at least three of ten service modes as well as all ELM commands. 
As previously mentioned service modes one and two are prepared in the OBDMiddleware package but since the project has already 
exceeded the time requirements for a bachelor thesis it seemed reasonable to omit those parts.

\section{Installation How-To}

Several steps have to be taken before starting or building the project on a development environment.
This project has a lot of dependencies, which are listed in the built package, and it requires kernel modules
This section elaborates on additional dependencies and building instructions depending on the purpose.

This project uses different open source API to realize the desired functionalities.
The following packages are included in the official Ubuntu repositories and can be downloaded by any package management system:

\begin{itemize}
 \item libmysqlcppconn-dev 
 \item mysql-server
 \item libusb 
 \item libusb-dev
 \item libxml++2.6-2 
 \item libxml++2.6-2-dev 
 \item qt-sdk
 \item libcppunit-dev 
 \item libcppunit-1.13-0 
 \item libboost-all-dev 
\end{itemize}

The only packages which require manual compilation are the libusb-vhci library and the vhci\_hcd package, 
resulting in two kernel module files (.ko). Those files either have to be inserted manually using insmod 
or they have to be installed with the help of make install. When installing and frequently accessing the 
kernel modules it is advisable to alter the modules system file, located under “/etc/modules”, by adding 
them at the end of the file. This will result in the loading of the kernel modules at startup.

To compile the libusb-vhci project the following steps have to be done:

\subsection{vhci\_hcd}

\begin{enumerate}
 \item download or use from repository
 \item make
 \item sudo make install
\end{enumerate}

Optional - depending on how kernel modules want to be handled:

\begin{enumerate}[resume]
 \item insmod usb-vhci-hcd.ko
 \item insmod usb-vhci-iocifc.ko
\end{enumerate}
 
or:
 
\begin{enumerate}
\setcounter{enumi}{3}
 \item sudo gedit /etc/modules \&
 \item add kernel module names (without extension) to file and save
\end{enumerate}

\subsection{libusb\_vhci [userspace tools]}

\begin{enumerate}
 \item download or use from repository
 \item ./configure
 \item make
 \item sudo make install
 \item sudo ldconfig -v 
\end{enumerate}

Depending on the Linux kernel version and the current vhci\_hcd and libusb\_vhci versions, 
the compiling procedure may vary. These instructions are extracted from the single INSTALL 
files from the repositories. The last step when installing is to tell the Linux ld path that 
libusb-vhci has been added to the library path so that it can be included normally.

\section{Building the Project}

When building the OBD project it is essential that it is built in the right order. 
The OBDMiddleware package  has to be built before the project specific projects. 
Each part of the OBDMiddleware package can be compiled separately, although it is not advisable, 
since the packages have internal dependencies. 
With the support of CMake building in any location is possible with the command:

\begin{verbatim}
 cmake <path to parent CMakeLists>
\end{verbatim}

Furthermore a package can be built using make package. This will result in a .deb file for Ubuntu distributions. 
Executing the generated file leads to the installation of the files into the library path of the Linux system. 

Like in most software projects it was deemed necessary, that the project can be built with just a compiled version 
of the middleware in case some changes emerge. Thus the project specific projects can be built without parameters 
if the OBDMiddleware package is installed. When intending to use the compiled version it can be built by delivering 
the CMake variable Middleware\_BASE\_PATH.

\begin{verbatim}
 cmake -DMiddleware_BASE_PATH=/home/<...>/OBDMiddleware/build/ .. 
\end{verbatim}

If the path has changed it is advisable to clear the build directory or at least delete the generated CMakeCache.txt 
since CMake has caching issues and does not reevaluate the given variables but rather uses the stored paths. 



%% vim:foldmethod=expr
%% vim:fde=getline(v\:lnum)=~'^%%%%\ .\\+'?'>1'\:'='
%%% Local Variables: 
%%% mode: latex
%%% mode: auto-fill
%%% mode: flyspell
%%% eval: (ispell-change-dictionary "en_US")
%%% TeX-master: "main"
%%% End: 

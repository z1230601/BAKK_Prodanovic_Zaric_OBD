%%%% Time-stamp: <2012-08-20 17:41:39 vk>

%% example text content
%% scrartcl and scrreprt starts with section, subsection, subsubsection, ...
%% scrbook starts with part (optional), chapter, section, ...
\chapter{Implementation}

The theory in the above described chapters was partially acquired during the research phase but mostly during the implementation of the packages. 
The basics in these chapters are not adjusted to the particular use in this project and cover also parts, which are implicitly used. 
Therefore this section focuses on the implementational details of the single packages and subpackages fitting the above described theory to the 
implementational approach.

\section{Serial Communication}

One of the main functionalities needed by both tools is the communication to the device attached to the USB port. Although this does not relate 
to OBD in particular, it is a common requirement in software projects which are either distributed (distributed systems) or  such which are 
meant for standardized distributions (Linux, Windows, Mac) in combination with external hardware. For this concrete example of this sort of 
implementation the ELM327 v1.4/v1.5 was used as external hardware. The goal of this middleware package is to provide development tools for easy 
communication with the ELM chip without specifics of port (USB, COM etc.). Another requirement which falls into this category is the emulation 
of a given USB ELM device to enable testing of OBD tools, leading to the implementation of a virtual USB FTDI device and controlling its behaviour.

The implementation of this package is based on the knowledge acquired in the chapter Universal Serial Bus (USB) and Command Structure and Behaviour.

\subsection{Serial Communication}

After acquiring an overview of the general OBD functionality and familiarizing with the ELM327 the next step was trying the terminal program 
Screen on Linux. It enables the user to send hexadecimal coded commands to any mounted device, in this case the ELM, and receives its responses, 
more concretely the diagnostic trouble code data. The screen starting command looks as follows

\begin{verbatim}
sudo screen /dev/ttyUSB0 <baud rate> 
\end{verbatim}

The ELMs communication baud rate is 38400 bits per seconds. Sending the ELM command ATZ causes the device to respond with its description.

%TODO: PICTURE

As visible on the image above Screen still has issues with carriage returns.

Naturally the next step was to try requesting trouble codes which was also successfully accomplished. An ignition coil failure simulation was 
caused by simply removing the power of one of the ignition coils of the tested vehicle. The requested temporary DTCs mirrored the anticipated 
ignition coil failure code as hexadecimal value. 

%TODO: PICTURE

The different open source libraries, like libusb, brought the capability of getting all devices and their information like vendor and product ID.
Furthermore a simple algorithm searching the sysfs of Linux provided the associated terminal file path (/dev/tty…). Last but not least the 
operations of reading and writing were achieved by the simple file descriptor system calls read and write in combination with termios for 
setting the baud rate of these operations. USB normally does not offer the ability of defining baud rates. This issue originated from the need 
of FTDI devices for RS232 ports. Therefore most external hardware, like Arduino, based on such chips, integrate a RS232 to USB converter 
requesting a communication baud rate. Additional research revealed the libftdi library, used exactly for these serial communication issues with 
USB. It combines functionalities of libusb with serial configurations like setting the baud rate. 

The usage of libftdi was momentarily discarded, due to no apparent benefit and the  additional disadvantage of refactoring the already existing 
code. The needed functionality was already covered by combining the libusb package with TTY Linux driver and file descriptors.

\subsection{Emulation}

The first approach consisted of exploiting the USBIP API, which provides the functionality of sharing USB devices over a network. After 
assessing its feasibility the decision was made to use the base library libusb-vhci.

The VHCI project can be described as a driver. VHCI (Virtual Host Controller Interface) is a kernel module with the capability of emulating a 
USB hardware device to the kernel. Kernel modules are code modules executed in the kernel space. They need to be compiled on each Linux kernel 
independently to match its structure. Furthermore the functionalities defined by kernel modules have to be executed with root privileges.

Although compiling the kernel module on each Linux distribution individually and executing the emulation as root can be rated as a disadvantage,
it is capable of freely emulating USB devices and using the full scope of USB features defined in the USB standard. 

The already existing example code of the project emulates a simple HID (Human Interface Device) which mostly covered the needs of the project. 
Nevertheless, gradually refactoring the example code led to an emulation of an FTDI device. The result is comprised of three classes 
representing the emulation:

\begin{itemize}
 \item USBEmulationSupervisor
 \item USBRequestHandler
 \item EmulatedDevice
\end{itemize}

The supervisor handles the general setup of the VHCI structure, including work requests and distribution of different USB functionalities as 
well as the request handler. The request handler’s responsibility lies with parsing the USB requests and storing them in an instance of 
“EmulatedDevice” and responding accordingly to the request. Currently only one emulated device is possible at a time. The emulated device holds 
the device-, configuration- and language descriptors as well as the function pointer defining the callback. This callback is meant for giving 
free possibility of defining what shall happen on a specific command. Especially ELM commands are of interest since OBD commands are 
standardised. The implementation covers only the most essential functionalities of the VHCI library.

The inconvenience of needing administrator rights discourages the utilization of the emulation. This issue is unavoidable, since libusb and 
libusb\_vhci both use functionalities, like claiming a hardware device, which require administrator rights. The fully functional software is 
intended to have its suid bit set once, therefore granting it administrator rights by default or at least ensuring that it has to be executed 
as administrator.

%TODO: PICTURE

\section{Databases}

Storing and managing data is bound to become an issue due to the enormous amount of data sets in in relation to OBD. Therefore a thoroughly 
implemented database interface cannot be avoided. For reasons of simplicity the base query language for the database MySQL was chosen. The main 
advantages leading to the choice of MySQL is the easy installation of MySQL server and client on all Linux distributions as well as the API 
availability. MySQLConnector++ is a C++ API for MySQL, which supplies the basic framework for executing and parsing responses of any database. 
Since it is an external library  Clean Code advises to wrap its API in a self defined interface, which is represented by this subpackage. 

The general structure composes of a class responsible for dispatching the requests, a class for parsing the data into a C++ conform class 
structure as well as an executor class defining the insertion and deletion commands. Keeping the philosophy of TDD (Test Driven Development) 
in mind it seemed wise to setup a test database as an SQL dump and supply a script reinitiating the database. After a testing function has 
manipulated the database this script can be used to reset the test database to its initial state. 

The configuration parameters can either be entered hardcoded or by parsing an XML file containing the host address, user, password and database 
name. The parsing  structure is thoroughly explained in the Configuration section. The XML looks as follows:

\begin{verbatim}
 <?xml version="1.0" encoding="UTF-8"?>
 <!DOCTYPE database SYSTEM "database.dtd">
 <database>
   <address>127.0.0.1</address>
   <user>obd</user>
   <password></password>
   <dbname>OBD_TroubleCodes</dbname>
 </database>
\end{verbatim}

The password field is left empty as the user obd has no password setup.  

%TODO: PICTURE

\section{OBD Base}

The OBDBase package represents the OBD specific part of this thesis. It provides links between the OBD structures like hexadecimal responses 
from the ELM to the database entries with their column definitions and IDs, as well as calculations, minimum and maximum values of single 
sensors and ELM specific command representations.

The OBD interface responds to PIDs of service mode I and II with a certain amount of bytes. This amount is defined in the ISO 15031 depending 
on the sensor the PID represents. PIDs that are modulo 32 are requests to get a 32 bit value of implemented PIDs of the next 32 PIDs. Meaning 
PID 0 (e.g. 01 00) returns four bytes where each bit is interpreted as a boolean value, stating the availability of the corresponding PID. The 
interpretation of these received bytes is defined in the above mentioned ISO standard as well. Analyzing the accessible PIDs a grouping into 
four different value types could be extracted.7

\subsection{Calculation Values}

Calculation values are responses or part of a response, where the received value represents the value of the specific PID’s underlying entity.
As the name suggests there are additional calculations necessary. The standard defines the environment parameters of these values, which are 
supplied by an XML file. PID 0x66 shows an example of a calculation value:

\begin{verbatim}
 <value interpretation="calculation">
  <name>MAF Sensor A</name>
  <bytes>2</bytes>
  <min>0</min>
  <max>2047.96875</max>
  <unit>g/s</unit>
 </value>
\end{verbatim}

These definitions describe the range of the received data, in this case from 0 - 2047.96875 g/s for 0 - 65535 decimal value of two bytes. 
This means that a received byte value of 0 corresponds to 0 g/s and a decimal value of 65535 represents 2047.9687 g/s. When resolving these 
relations the following formula can be used for converting values: 

\[ c_s = \frac{max - min}{2^{bits} - 1} \]

\[ v_{interpreted} = v_{raw} * c_s + min \]

\begin{verbatim}
 void OBDCalculationValue::interpretCalculationValue(std::vector<uint8_t> input)
 {
   unsigned int compound_value = calculateCompoundValue(input);
   double scale = (max_ - min_) / (pow(2.0, byte_amount_ * 8.0) - 1);
   interpreted_value_ = compound_value * scale + min_;
   uninterpreted_value_ = compound_value;
 }
\end{verbatim}

The conversion of interpreted values to raw values  naturally deduces:

\[ v_{raw} = \frac{(v_{interpreted} - min)}{c_s} \]

\begin{verbatim}
 void OBDCalculationValue::interpretCalculationByteArray(double value)
 {
   interpreted_value_ = value;
   interpreted_value_ = std::min(interpreted_value_, max_);
   interpreted_value_ = std::max(interpreted_value_, min_);
   double scale = (max_ - min_) / (pow(2.0, byte_amount_ * 8.0) - 1);
   uninterpreted_value_ = (interpreted_value_ - min_) / scale;
 }
\end{verbatim}

\subsection{Value Mapping Values}

These are values whose raw values are used on a predefined mapping to get a string representation of this specific PID’s request. Meaning the 
value directly maps to a result string. The configuration of such a value requires only the mapping itself, as illustrated by the PID 0x5f:

\begin{verbatim}
 <value interpretation="mapping">
  <bytes>1</bytes>
  <mapping type="value">
   <entry from="0E">LKW (Euro IV) B1</entry>
   <entry from="0F">LKW (Euro V) B2</entry>
   <entry from="10">LKW (Euro EEV) C</entry>
  </mapping>        
</value>
\end{verbatim}

\subsection{Bit Mapping Values}

As the name suggests bit mapping values are values, where each bit has a different meaning. Information required from the ISO standard is only
the mapping and significance of the single bits. In some cases a bits true and false value signify different outputs for the same entity. An 
example of such a value is PID 0 as well as PID 0x65.

\begin{verbatim}
 <value interpretation="mapping">
  <bytes>1</bytes>
  <mapping type="bit">
   <entry from="0" set="false">Kraftaufnahme inaktiv</entry>
   <entry from="0" set="true">Kraftaufnahme aktiv</entry>
   <entry from="1" set="false">Automatikgetriebe in Park-/Neutralstellung</entry>
   <entry from="1" set="true">Vorwärts- oder Rückwärtsgang</entry>
   <entry from="2" set="false">Manuelles Getriebe in … Kupplung getreten</entry>
   <entry from="2" set="true">Gang eingelegt</entry>
   <entry from="3" set="false">Vorglühlampe aus</entry>
   <entry from="3" set="true">Lampe ein</entry>
  </mapping>
 </value>
\end{verbatim}

The from field specifies the bit position while the set attribute is an optional attribute to assign a bits value to a concrete output. In this 
case it is supplied, meaning that depending  on the bits status different strings are mapped. If the set attribute is not explicitly stated it 
is assumed true.

\begin{verbatim}
 std::string OBDBitMappingValue::interpretToValue(std::vector<uint8_t> input)
 {
   interpreted_value_ = calculateCompoundValue(input);
   uninterpreted_value_ = interpreted_value_;
   return getInterpretedValueAsString();
 }
\end{verbatim}

As this code shows in case of mapping values of interpreted and uninterpreted values are identical and represent the key of the underlying mapping.

\subsection{Bit Combination Value Mapping}

This value parsing is a bit comparator mapping of more than two bits. The from field specifies the bit positions and the set represents the 
value it needs to equal to get the result string in question. PID 0x70 is an example of such a mapping.

\begin{verbatim}
 <value interpretation="mapping">
  <bytes>1</bytes>
  <mapping type="bitcombination">
   <validitybit from="01">2</validitybit>
   <entry from="01" set="01">Offener Kreislauf, kein Fehler</entry>
   <entry from="01" set="10">Geschlossener Kreislauf, kein Fehler</entry>
   <entry from="01" set="11">Fehler vorhanden, Wert unzuverlässig</entry>
   <validitybit from="23">5</validitybit>
   <entry from="23" set="01">Offener Kreislauf, kein Fehler</entry>
   <entry from="23" set="10">Geschlossener Kreislauf, kein Fehler</entry>
   <entry from="23" set="11">Fehler vorhanden, Wert unzuverlässig</entry>
    </mapping>    
 </value>
\end{verbatim}

\subsection{Inner Workings}

Since most PIDs do not consist of only one value or type, values have to be defined in the order they are supposed to be received. They can be 
combined as needed, so that multiple calculation values can be followed by any mapping value or vice versa. The total of the single value 
definition’s bytes will then be expected when parsing the OBD commands answer, giving significance to the order of the sequence in which the 
values are defined in the XML file.

\begin{verbatim}
 <obdcommand>
  <pid>70</pid>
  <description>Ladedruckregelung</description>
  <validitymapping  mode="manual">true</validitymapping>
  <values>
   <value interpretation="calculation">
    <name>Soll Ladedruck A</name>
    <bytes>2</bytes>
    <min>0</min>
    <max>2047.96875</max>
    <unit>kPa</unit>
    <validitybit>0</validitybit>    
   </value>
   <value interpretation="calculation">
    <name>Ist Ladedruck A</name>
    <bytes>2</bytes>
    <min>0</min>
    <max>2047.96875</max>
    <unit>kPa</unit>    
    <validitybit>1</validitybit>            
   </value>
    .
    .
    .
   <value interpretation="mapping">
    <bytes>1</bytes>
    <mapping type="bitcombination">
     <validitybit from="01">2</validitybit>
     <entry from="01" set="01">Offener Kreislauf, kein Fehler</entry>
     <entry from="01" set="10">Geschlossener Kreislauf, kein Fehler</entry>
     <entry from="01" set="11">Fehler vorhanden, Wert unzuverlässig</entry>
     <validitybit from="23">5</validitybit>
     <entry from="23" set="01">Offener Kreislauf, kein Fehler</entry>
     <entry from="23" set="10">Geschlossener Kreislauf, kein Fehler</entry>
     <entry from="23" set="11">Fehler vorhanden, Wert unzuverlässig</entry>
    </mapping>        
   </value>
  </values>
 </obdcommand>
\end{verbatim}

Furthermore special OBD commands exist where the first byte defines which of the transmitted following bytes are valid values. These OBD 
commands are mostly used for array sensor PIDs. The package offers therefore the optional definition of an automatic or manual validity mapping.

A validity mapping contains one or more bytes beginning the response in which the position of the bits correspond to the following bytes and 
their value. The value on a certain position in this mapping byte indicates if the transmitted value is valid or can be ignored.

There are no rules without exceptions. Certain values have a complex validity mapping structure especially in association with the bit 
combination mapping. Therefore the need for manually configurable validity mapping emerged which got implemented as the manual validity mapping 
mode. The amount of such values is limited (one or two occurrences). In this mode the validity bit position has to be defined separately with a 
tag in the all corresponding value definition. 

High abstraction is achieved with an abstract base class. It defines main utility functions like adding bytes to a compound value, storing the 
raw and interpreted data and delegating the interpretation trough a pure virtual function to its derived classes. 

A simple factory pattern which creates OBDCommandValues from its XML representation ensures easy usability. It creates the values depending on 
the interpretation attribute of each value tags definition and returns an abstract object representing the value. 

\subsection{ELM Commands}

There is a set of ELM specific commands configurable for the communication to the control unit of the automobile. Generally the emulation has to 
answer those requests and the OBD tool must configure the ELM specific settings. Therefore it seemed necessary to implement a parsing and object 
representation of the ElmCommands for our middleware package.

The difficulty of this part of the OBDBase package surfaced when inspecting the single commands more closely. The datasheet of the ELM offered 
an inconsistent way of defining additional values. In most cases they are referenced as “h” (hexadecimal value) but there are some values 
referenced by “x”,” y” or “z” as well.

\begin{verbatim}
 <command>
  <version>1.0</version>
  <elmcommand>"AL"</elmcommand>
  <description>"Allow Long (>7 byte) messages"</description> 
  <group>"OBD"</group>
 </command>
\end{verbatim}

\section{Configuration}

The configuration subpackage is used to parse and write XML files. Currently there are following configuration files: 

\begin{itemize}
 \item elmcommandconfiguration.xml
 \item obdcommand.xml
 \item obdcuConfiguration.xml
 \item obdtoolConfiguration.xml
 \item dbconfiguration.xml
\end{itemize}

These contain configuration parameters for the software. The command configurations (elmcommandconfiguration.xml and odbcommand.xml) contain all 
commands known by the ELM327 extracted from its datasheet or all OBD commands consisting of service ID with its corresponding PID and name. 
The project configurations (obdcuConfiguration.xml and obdtoolConfiguration.xml) are only path declarations to the other configurations enabling 
factory resets.

This package uses the functionalities of the libxml++ library. This library uses the DOM (Document Object Model) and supplies functionalities 
for parsing an XML file into a DOM model as well as writing a DOM model to a file. 

The base of the configuration package consists of the XMLWriter, XMLReader and DefaultHandler classes. Their purpose is to parse a DOM object 
into internal class representations, as suggested in Clean Code, therefore simplifying the usage of the libxml++ library functions to creating 
a derived class of the DefaultXMLHandler. 

This class has a pure virtual method handleNode which gets a xmlpp::Node* as parameter. A derived class instance can distinguish its execution 
and parsing behavior by comparing e.g. the name of the xmlpp::Node. The DefaultXMLHandler supplies protected functions for parsing 
xmlpp::TextNode* into C++ strings or setting xmlpp::TextNode* from C++ strings. 

Handlers only define the handling of the nodes, more precisely setting  or parsing the underlying libxml++ DOM representation (xmlpp::Document). 
The XMLWriter and XMLReader classes are initialized with an instance of the DefaultXMLHandler. Its parse or write function requires an already 
existing XML file. Each function calls its internally defined DefaultHandler’s handleNode function while recursively iterating through DOM nodes. 
A well formed structure of the underlying XML files is ensured by the required DTD files, shown beneath, defining the structure of the XML files.

\begin{verbatim}
 <?xml version="1.0" encoding="UTF-8"?>
 <!DOCTYPE database SYSTEM "database.dtd">

 <database>
  <address>127.0.0.1</address>
  <user>obd</user>
  <password></password>
  <dbname>OBD_TroubleCodes</dbname>
 </database>
\end{verbatim}

\begin{verbatim}
 <!ELEMENT database (address, user, password, dbname)>
 <!ELEMENT address (#PCDATA)>
 <!ELEMENT user (#PCDATA)>
 <!ELEMENT password (#PCDATA)>
 <!ELEMENT dbname (#PCDATA)>
\end{verbatim}

The difference between the two classes is that, whilst the XMLReader only iterates and lets the handler save the read data, the XMLWriter sets 
the DefaultHandler into a mode, where he instead of parsing the node’s data sets it. After the iteration the XMLWriter writes to the same file 
it read from with the newly filled data.
% TODO: PICTURE
%% vim:foldmethod=expr
%% vim:fde=getline(v\:lnum)=~'^%%%%\ .\\+'?'>1'\:'='
%%% Local Variables: 
%%% mode: latex
%%% mode: auto-fill
%%% mode: flyspell
%%% eval: (ispell-change-dictionary "en_US")
%%% TeX-master: "main"
%%% End: 
%%%% Time-stamp: <2012-08-20 17:41:39 vk>

%% example text content
%% scrartcl and scrreprt starts with section, subsection, subsubsection, ...
%% scrbook starts with part (optional), chapter, section, ...
\chapter{Motivation}

The motivation for this project is the increasing complexity of modern car systems. The amount of integrated control units in automotive 
systems these days tends to go beyond 80. The resulting energy consumption for the electrical components in such a system sums up to 2,5 kW. 
This grounds the decision of the automotive industry to deal with its own energy management of these components. Considering the number of 
control units and their energy consumption an effective handling of these units seems inevitable. Since the knowledge about the fundamental 
functionality of control units is kept with the manufacturer, serious knowledge deficit surfaces regarding control units from the perspective 
of a common service mechanic. Thus leading to a dependency on expensive information and/or tools provided by the manufacturer of the automobile.
The car service shop is forced to rely on those tools. If suspicion of a fault in a control unit arises it is common practice to change the whole 
unit to resolve this suspicion rather than fixing the real cause of the actual error.
On the other hand this opens possibilities for the manufacturer to keep his income over development periods high. Since knowledge about the 
specifics or the right protocols is restricted communication with the full functionality extent giving automotive manufacturers the chance to 
sell those protocols or a service for diagnosing, keeping their income high.
After discovering that there are standards defining most of the diagnostic trouble codes (DTCs) and the possibility to request those with 
hardware available at an economic price it seems reasonable to develop an open source software implementing those functionalities. Further 
than just requesting DTCs the software is meant to entail ELM327 features as well. 

\section{Market Analysis of Existing OBD Tools}
The first impression of the OBD diagnostic tools market may seem vast, but focusing on open source solutions it is limited to a handful of tools.
During the research phase we found different software starting by pyOBD and ScanTool, over ScanMaster and EasyObd up to OBD Auto Doctor. 
When putting them under practical tests huge difficulties emerged. The following analysis is meant to provide the foundation for this bachelor 
thesis to implement an open source OBD tool as well as a test system for OBD tools under Linux, which should easily be extendable to Windows.

The strategy on keeping the software easily extendable is to adapt the general programming style to use system independent libraries, thus 
ensuring the ability to run on most common systems. As a foundation of the market analysis testing the currently available software seem reasonable. 
Therefore the basic testing strategy is to choose a distribution and try to execute the following steps:

%ENUMARTION
establish a connection to the OBD interface via an ELM327 microcontroller 
read trouble codes produced by self-made errors
reset trouble code memory

Those steps define the basic functionalities in every OBD standard which can be considered as the minimum requirements for an OBD diagnosis tool.
By acquiring two standard USB to OBD cables at least hardware independence can be established to a certain degree. Additionally the conclusion, that 
the costs of the required hardware are low, can be drawn and thus leading to average consumers with low budgets representing the target audience. Due
to the requirement of hardware independence the focus of the project shifts to a pure software implementation. Even though only the use of hardware 
based on the ELM327 chip is intended, compatibility for most ELM chips is the long term goal. 

Taking the variety of the target audience into account, as many different cars and systems as available served as testing environments to experience 
the behaviour of the tested tools under different circumstances. Those cars consist of a Renault Trafic 2006 1.9 TDI, an Opel Vivaro 2009, 
an Opel Zafira 2004 as well as a VW Sharan 2003 1.9 TDI. By using Windows and Ubuntu the majority of distributions are covered.

\subsection{Windows}
Four different OBD softwares on a Windows 7 OS are tested:
%ENUMARTION
EasyObdII Version 2.5
OBD Auto Doctor
ScanMaster-ELM DEMO
OBD 2007

With OBD 2007 and EasyObdII Version 2.5 problems occurred while just trying to connect to the control unit, making it difficult to conduct the tests. 
Different attempts to solve this problem failed.\\
Since only trial/demo versions are available for free, the range of testable functions is limited to measuring the actual engine RPM, the vehicle 
speed or the load value. Due to the encountered problems, a satisfying analysis could not be executed thoroughly.
\subsection{Ubuntu}
Connecting to the control unit under Linux proved to be a problem as the ELM327 uses the USB port (ttyUSB) but most of the tested software only 
searches for COM ports (ttyS).
Assuming intact hardware, no immediate solution appeared of how to mount the USB device onto the COM port nor how to configure the software such that 
it connects to the USB port. However, using the low level standard software screen supplied by Linux seemed to be the best approach, as it enables 
communication with any serial device in form of HEX commands. With this approach the request for trouble codes in HEX format is possible, but 
additional information such as datasheets, computer knowledge as well as forums for decoding the response are necessary.

\subsection{Conclusion}
Although spending much time in trying to enable the software’s functionality on both systems, the benefit is limited. Judging from the interface of 
those freeware tools the usability is not satisfactory. Nevertheless, communicating with the ELM327 seems possible with the terminal based software 
used. In conclusion it can be stated that the functionalities of the device itself seem intact, though appropriate open source software is hard to 
find.
%% vim:foldmethod=expr
%% vim:fde=getline(v\:lnum)=~'^%%%%\ .\\+'?'>1'\:'='
%%% Local Variables: 
%%% mode: latex
%%% mode: auto-fill
%%% mode: flyspell
%%% eval: (ispell-change-dictionary "en_US")
%%% TeX-master: "main"
%%% End: 

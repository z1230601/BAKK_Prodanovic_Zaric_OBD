%%%% Time-stamp: <2012-08-20 17:41:39 vk>

%% example text content
%% scrartcl and scrreprt starts with section, subsection, subsubsection, ...
%% scrbook starts with part (optional), chapter, section, ...
\chapter{Motivation}
\label{sec:motivation} 

Since its inception control units are steadily increasing in complexity and amount of modern car systems. Nowadays it is not uncommon to 
integrate $80$ and more units in a single car. The resulting energy consumption for the electrical components in such a system sums up to $2.5$ kW.
This substantiates the decision of the automotive industry to establish an energy management for all electrical components. Considering this an 
effective handling of these units seems inevitable \cite{SCHAFOBD1}.

Since the manufacturers do not share the knowledge about the fundamental 
functionality of their control units, serious deficits surface regarding control units from the perspective 
of a common service mechanic. Thus leading to a dependency on expensive information and/or tools provided by the manufacturer of the automobile.
The car service shop is forced to rely on those tools. If suspicion of a fault in a control unit arises it is common practice to change the whole 
unit rather than fixing the real cause of the actual error. Furthermore wrong assumptions of the maintanence personnel often lead to incorrect 
error identification and thus the costly replacement of fully functional parts.

Correspondingly manufacturer thrive by offering spare parts, including control units, and keep their income over development periods high. The 
protocols needed to realize a diagnosis tool with the fullest possible extent of functionality are restricted. By protecting those, they seize the 
opportunity to sell those protocols or a service for diagnosing implementing them. 

The financial burden carried by common car holder motivated this thesis. Utilizing the publicly accessible standards \cite{ISO15031} defining most of the diagnostic 
trouble codes (DTCs) in combination with the available, low-priced hardware (ELM$327$) it is the goal of this thesis to supply a basis for development of 
an open source software implementing those functionalities.

\section{Market Analysis of Existing OBD Tools}
The first impression of the OBD diagnostic tools market may seem vast, but focusing on open source solutions it is limited to a handful of tools.
Deeper research revealed different software solutions as pyOBD, ScanTool, ScanMaster, EasyObd as well as OBD Auto Doctor. 
Practical tests resulted in many different difficulties. The following analysis is meant to provide the foundation for this bachelor 
thesis. 
% to implement an open source OBD tool as well as a test system for OBD tools under Linux, which should easily be extendable to Windows.

The strategy on keeping the software easily extendable is to use system independent libraries, thus 
ensuring the ability to run on most common systems. As a foundation of the market analysis testing the currently available software seems reasonable. 
Therefore the basic testing strategy is to choose a distribution and try to execute the following steps:

%ENUMARTION
\begin{enumerate}
 \item establish a connection to the OBD interface via an ELM$327$ microcontroller 
 \item read trouble codes produced by self-made errors
 \item reset trouble code memory
\end{enumerate}

Since those steps are minimum requirements extracted from basic functionalities described in each OBD standard, they are applicable as minimum 
requiremets for OBD tools as well. 

By acquiring two common USB to OBD cables at least a certain degree of hardware independency can be established. 
Additionally the conclusion, that the costs of the required hardware are low, can be drawn and thus leading to average consumers with low budgets 
representing the target audience. Due to the requirement of hardware independency the focus of the project shifts to a pure software implementation. 
Even though only the use of hardware based on the ELM$327$ chip is intended, compatibility for most ELM chips is the long term goal. 

Taking the variety of the target audience into account, as many different cars and systems as available served as testing environments to experience 
the behaviour of the tested tools under different circumstances. Those cars consist of a Renault Trafic $2006$ $1.9$ TDI, an Opel Vivaro $2009$, 
an Opel Zafira $2004$ as well as a VW Sharan $2003$ $1.9$ TDI. By using Windows and Ubuntu the majority of distributions are covered.

\subsection{Windows}
Four different OBD softwares on a Windows $7$ OS are tested:
%ENUMARTION

\begin{itemize}
 \item EasyObdII Version $2.5$
 \item OBD Auto Doctor
 \item ScanMaster-ELM DEMO
 \item OBD $2007$
\end{itemize}

OBD $2007$ and EasyObdII Version $2.5$ did not connect to the control unit, making it difficult to conduct the tests. 
Different attempts to solve this problem failed.

Since only trial versions are available for free, the range of testable functions is limited to measuring the current engine RPM, the vehicle 
speed or the load value. Due to the encountered problems, a satisfying analysis could not be executed thoroughly.

\subsection{Ubuntu}
Connecting to the control unit under Linux proved to be a problem as the ELM$327$ uses the USB port (ttyUSB) but most of the tested software only 
searches for COM ports (ttyS).
Assuming intact hardware, no immediate solution appeared of how to mount the USB device onto the COM port nor how to configure the software such that 
it connects to the USB port. However, using the low level standard software screen supplied by Linux seemed to be the best approach, as it enables 
communication with any serial device in form of HEX commands. With this approach the request for trouble codes in HEX format is possible, but 
additional information such as datasheets, computer knowledge as well as forums for decoding the response are necessary.

\subsection{Analysis}
Although spending much time in trying to enable the software’s functionality on both systems, the success is limited. Judging from the interface of 
those freeware tools the usability is not satisfactory. Nevertheless, communicating with the ELM$327$ using the terminal based software succeeded. 
In conclusion it can be stated that the functionalities of the device itself seem intact, though appropriate open source software is hard to 
find.

\section{Goals}

The previous section surfaced the problem that there is no sufficient
way of testing OBD software without spending money mostly on
licensing. This leads to the separation of the project
into three parts.

Firstly an analysis of the OBD system and standards has to be conducted, focusing on actual capabilities and security, as well as needed interfaces 
like USB and software development techniques.The first three can be found in the chapter \nameref{sec:OBDBASICS}. The latter stands apart in the chapter 
\nameref{sec:SDT}, where the theory and our experience and approaches are summarized.

The seperation seems reasonable due to the nature of the content of chapter \nameref{sec:SDT} 
, which describes the engineering process in more detail. Software development in the automotive context has different, more
strict, approaches than software developed for non critical systems.
At this point it is important to highlight that both processes suggest
testing as an essential procedure. Taking this into
consideration the main strategy is to provide an emulated ECU first,
which will serve as a testing environment, and then implement the tool
itself.

The capabilities of the ECU testing environment are derived from the
OBD standard, e.g. it should be possible to set different trouble codes,
generic as well as manufacturer specific. The following enumeration is
a rough outlining of the capabilities it should have:

\vbox{
\begin{description}
\item[MUST]   set/unset trouble codes
\item[MUST]   define trouble codes
\item[MUST]   online capability
\item[MUST]   mount virtual OBD-Tool
\item[CAN]    server capabilities (not obligatory)
\item[SHOULD] simulate sensor data in operating condition (given models;
  adjustable)
\end{description}}

A more detailed requirements documentation is attached in the \nameref{sec:Appendix}.

Finally implementing an open source OBD tool with all its
functionalities provided by the OBD standard is
intended. In contradiction to existing and already tested tools usability and easy
connectivity will have high priority.

An optional goal for the implementation process is to extend the range
of fault codes detectable by our software with a separate (online)
database.
%% vim:foldmethod=expr
%% vim:fde=getline(v\:lnum)=~'^%%%%\ .\\+'?'>1'\:'='
%%% Local Variables: 
%%% mode: latex
%%% mode: auto-fill
%%% mode: flyspell
%%% eval: (ispell-change-dictionary "en_US")
%%% TeX-master: "main"
%%% End: 

%%%% Time-stamp: <2012-08-20 17:41:39 vk>

%% example text content
%% scrartcl and scrreprt starts with section, subsection, subsubsection, ...
%% scrbook starts with part (optional), chapter, section, ...
\chapter{Goals}

The previous section surfaced the problem that there is no sufficient
way of testing OBD software without spending money mostly on
licensing. This is leading to the conclusion that the project splits
into three parts.

Firstly an analysis of the OBD system and standards has to be
conducted, focusing on actual capabilities and security. As our
research revealed the OBD interface has direct access to the
diagnostic control unit and therefore to each bus used in the
automobile. This issue was one of the most important disadvantages
critics discussed since the introduction of
OBD.

As described in the chapter REF TO SOFTWAREDEVELOPMENT!!!!! the engineering process is discussed in more detail, as
software development in the automotive context has different, more
strict, approaches than software developed for non critical systems.
At this point it is important to highlight that both processes suggest
testing as an essential procedure. Taking this into
consideration the main strategy is to provide an emulated ECU first,
which will serve as a testing environment, and then implement the tool
itself.

The capabilities of the ECU testing environment are derived from the
OBD standard, e.g. it should be possible to set different trouble codes,
generic as well as manufacturer specific. The following enumeration is
a rough outlining of the capabilities it should have:

\begin{description}
\item[MUST] set/unset trouble codes
\item[MUST] define trouble codes
\item[MUST] online capability
\item[MUST] mount virtual OBD-Tool
\item[CAN] server capabilities (not obligatory)
\item[SHOULD] simulate sensor data in operating condition (given models;
  adjustable)
\end{description}

More detailed requirements can be found in the attached requirements
document.

Finally implementing an open source OBD tool with all its
functionalities provided by the OBD standard is
intended. In contradiction to existing and already tested tools usability and easy
connectivity will have high priority.

An optional goal for the implementation process is to extend the range
of fault codes detectable by our software with a separate (online)
database.

%% vim:foldmethod=expr
%% vim:fde=getline(v\:lnum)=~'^%%%%\ .\\+'?'>1'\:'='
%%% Local Variables: 
%%% mode: latex
%%% mode: auto-fill
%%% mode: flyspell
%%% eval: (ispell-change-dictionary "en_US")
%%% TeX-master: "main"
%%% End: 
